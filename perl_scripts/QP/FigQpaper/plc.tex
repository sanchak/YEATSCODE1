1
The natural function of phospholipases C may be to secure supplies of phosphate, and the regulation of some phospholipase C genes by exogenous phosphate levels supports this hypothesis.

  ================================= Gram +ve PC ===========================================================================
surprisingly, the B. cereus PC-PLC has been reported to show antigenic similarity with eukaryotic phospholipase C 

The C. perfringens alpha-toxin and B. cereus PC-PLC are the most intensively studied
C perfringens enzyme was a potent toxin (96, 97, 105) with hemolytic (139, 166), lethal (162, 168), dermonecrotic (100), vascular permeabilization (158), and platelet-aggregating (114, 157) properties

B. cereus PC-PLC and L. monocytogenes PLC-B are posttranslationally activated by the removal of 14 (70) or 26 (179) N-terminal amino acids, respectively.


 C. perfringens alpha-toxin and C. bifermentans PLC possess an additional C-terminal domain 

  This suggestion has recently been proven for the C. perfringens alphatoxin, because a truncated form of the protein, corresponding to the B. cereus PC-PLC, retained the phospholipase C activity but showed markedly reduced sphingomyelinase, hemolytic, and lethal activities (168).

modification of tyrosine residues abolishes hemolytic, lethal, and platelet-aggregating properties of the C. perfringens alphatoxin (140). Perhaps significantly, the reported homology between the C-terminal domain of the C. perfringens alphatoxin and the N terminus of arachidonate-5-lipoxygenase, a eukaryotic lipid-metabolizing enzyme, includes five aligned tyrosine residues (168). It seems possible that these hydrophobic tyrosine residues are similarly involved in the recognition of hydrocarbon substrates.

The protein-stabilizing effect of the zinc ions may account for the remarkable thermal stability reported for the B. cereus PC-PLC and C. perfringens alpha-toxin; either enzyme can survive heating to 100°C for short periods (123, 162)

=======================================================Gram +ve PI =========================================================================
 An important feature of the PI-PLCs is their ability to cleave the phosphatidylinositol-glycan-ethanolamine anchor to which many eukaryotic membrane proteins are attached

=======================================================Gram -ve PC =========================================================================
o The phospholipases C produced by P. aeruginosa 
 the insertional inactivation of the P. aeruginosa hemolytic phospholipase C (PLC-H) did not completely abolish phospholipase C activity and thereby identified a second, nonhemolytic phospholipase C (PLC-N) produced by this bacterium

 PLC-H is posttranslationally modified, by one of the pkcR gene products via an unknown mechanism, to yield a product with altered charge and greater hemolytic activity
  pIcS, which encodes the PLC-H enzyme, and plcN, which encodes PLC-N, are distally located on the chromosome (118), and the encoded proteins are of similar molecular weights, but whereas PLC-N is a basic protein (pI 8.8), PLC-H is acidic (pI 5.5) (118). 
  Since both enzymes are able to digest phosphatidylcholine but only PLC-H can digest sphingomyelin, this difference in substrate specificity must be due to differences in recognition of the hydrocarbon tails rather than of the head group (which is identical).

 Induction of PLC-H was independent of concentration and PhoB, whereas PLC-N induction was seen only under low-Pi conditions and required PhoB 

 ==============================================================================================================================================

  are induced under low-phosphate conditions, and regulation appears to be at the transcriptional level 

 The outer leaflet is made up mainly of phosphatidylcholine and sphingomyelin 
  The C perfringens alpha-toxin and P. aeruginosa PLC-H hydrolyze phosphatidylcholine and sphingomyelin and are hemolytic, whereas the structurally related nonhaemolytic B. cereus PC-PLC and P. aeruginosa PLC-N are not able to effectively hydrolyze sphingomyelin. Additional evidence for this hypothesis is provided by reports that the B. cereus sphingomyelinase and PC-PLC can act synergistically to cause hemolysis (56)

 Only some phospholipases C have been reported to be hemolytic and lethal and to have necrotizing activities. However, it is now apparent that nonhemolytic phospholipases C can act in conjunction with other proteins to cause cell lysis. 

Several lines of evidence suggest that phospholipases C are able to stimulate the arachidonic acid cascade in cells.

 It seems possible that the generation of diacylglycerol by exogenously applied bacterial phospholipases C mimics the effects of normal eukaryotic cell enzymes, where the generated diacylglycerol serves as a secondary messenger 

 The molecular basis of these effects has not been fully elucidated, but it is known that protein kinase C (PKC) can be activated by diacyglycerol and/or increased intracellular calcium levels (15). Thus it seems possible that bacterial phospholipases C activate PKC via the generation of diacylglycerol. Since it has been suggested that PKC can activate the eukaryotic cell phospholipases C and D, this pathway would serve as a positive-feedback loop.	


 lthough bacterial phosphatidylinositol-hydrolyzing phospholipases C have not been shown to increase the levels of inositol triphosphate in eukaryotic cells, there is no reason why this effect could not be elicited, especially if the phospholipase C could gain access to the inner leaflet of the plasma membrane.


 he phosphatidylinositol-glycanethanolamine anchor can be cleaved by membrane-active phosphatidylinositol-specific phospholipases C (75) to release a variety of cell membrane-bound enzymes (Table 6). The reported increase in blood alkaline phosphatase levels following intravenous administration of the B. cereus PlPLC (66) suggested that these enzymes elicit similar effects in vivo and in vitro. The significance of these released enzymes has not been identified, but it seems possible that the eukaryotic alkaline phosphatase could be used by the bacterium as part of a phosphate-scavenging pathway. This would be especially significant since, as noted above (66), the level of Pi in the blood is below that required for the growth of many bacteria


2.
In general, secreted phospholipases are thought to function in phosphate acquisition, carbon source acquisition, and in some cases as virulence factors for pathogenic species.

phospholipid substrates are defined by their polar head group, of which the most common in mammalian cells are phosphatidylcholine (PC), phosphatidylserine (PS), phosphatidylinositol (PI), and phosphatidylethanolamine (PE)

In contrast to mammalian cells, E. coli membranes are primarily composed of PE with some phosphatidylglycerol, cardiolipin (double phospholipid joined at the phosphate by glycerol) and trace amounts of PS. Furthermore, only a subset of the products of phospholipase activity are direct precursors for second messengers in mammalian cells, 

Two major divisions of phospholipase activities can be defined by the site of cleavage, whether the cleavage is in the hydrophobic diacylglycerol moiety (PLA) or in the polar head group of the amphipathic phospholipid (PLC and PLD).


Historically, the first indication that bacterial phospholipases were virulence factors was the realization that some bacterial toxins were in fact secreted phospholipases, for example the Clostridium perfringens α toxin [7] and Staphylococcus aureus toxin [8].


In general, phospholipase toxicity has been linked to cytolytic activity and is presumed to be directly due to phospholipase activity upon membrane phospholipids and membrane destruction.

i even the two leaflets (inner and outer) of a particular membrane may have very distinct phospholipid composition. For instance, the plasma membrane of most eukaryotic cells contains predominantly PC and sphingomyelin in the outer leaflet and PI, PE, and PS in the inner leaflet. Therefore, an extracellular phospholipase would not be expected to lyse a target cell unless the outer leaflet phospholipids were efficiently hydrolyzed substrates of the enzyme. Of course, this does not exclude possible cooperativity between several phospholipases or between a phospholipase and another type of ‘lysin’ (hemolysin or cytolysin) molecule as has been proposed for Listeria monocytogenes 

o Cytolysis is one of the more common characteristics attributed to bacterial phospholipase virulence factors (table I). Whether cytolysis results from the accumulation of membrane destabilizing products or by the wholesale destruction of membrane phospholipids, it can be caused directly by a very active bacterial phospholipase with broad specificity or in concert with host degradative enzymes induced by the bacterial phospholipase. Naturally, the cytolytic activity varies greatly amongst bacterial phospholipases.

Immune modulation seems to be a fairly common trait amongst those phospholipase virulence factors which have been more thoroughly studied. Perhaps this effect is not altogether surprising if one considers that addition of an exogenous phospholipase activity (i.e., bacterial) is likely to upset the delicate control designed to stimulate an adequate, but not overly damaging, immune response. 



The roles that bacterial phospholipases play in disease is quite varied – from triggering bacterial entry, endosomal lysis, and cytolysis to modulating the local immune response and stimulating cytokine secretion.



3

Phosphoinositide-specific phospholipases C (PI-PLCs) are ubiquitous enzymes that catalyse the hydrolysis of phosphoinositides to inositol phosphates and diacylglycerol (DAG). Whereas the eukaryotic PI-PLCs play a central role in most signal transduction cascades by producing two second messengers, inositol-1,4,5-trisphosphate and DAG, prokaryotic PI-PLCs are of interest because they act as virulence factors in some pathogenic bacteria. Bacterial PI-PLCs consist of a single domain of 30 to 35 kDa, while the much larger eukaryotic enzymes (85 to 150 kDa) are organized in several distinct domains. The catalytic domain of eukaryotic PI-PLCs is assembled from two highly conserved polypeptide stretches, called regions X and Y, that are separated by a divergent linker sequence. There is only marginal sequence similarity between the catalytic domain of eukaryotic and prokaryotic PI-PLCs. 



A striking difference between the enzymes is the utilization of a catalytic calcium ion that electrostatically stabilizes the transition state in eukaryotic enzymes, whereas this role is ®lled by an analogously positioned arginine in bacterial PI-PLCs. The catalytic domains of all PI-PLCs may share not only a common fold but also a similar catalytic mechanism utilizing general base/acid catalysis. The conservation of the topology and parts of the active site suggests a divergent evolution from a common ancestral protein.


Phosphoinositide-speci®c phospholipases C (PIPLC) are ubiquitous enzymes that catalyse the speci®c cleavage of the phosphodiester bond of phosphoinositides (PI, PIP, PIP2) to generate watersoluble inositol phosphates (InsP, InsP2, InsP3) and membrane-bound diacylglycerol (DAG; Figure 1). In higher eukaryotes, PI-PLCs are key enzymes in most receptor-mediated signal transduction pathways. They catalyse the hydrolysis of phosphatidylinositol 4,5-bisphosphate (PIP2) to generate two second messengers, inositol-1,4,5-trisphosphate (InsP3) and DAG. InsP3 is released into the cytoplasm and results in in¯ux of Ca2<87> from internal stores, whereas DAG remains membrane resident and stimulates protein kinase C isozymes (Nishizuka, 1992; Berridge, 1993). Mammalian PIPLCs have been classi®ed into three families based on their primary structure and mode of activation: the b-family (150 kDa) is activated by association with heterotrimeric G-protein subunits, the gfamily (145 kDa) is activated by association with tyrosine kinases, and the simpler d-family (85 kDa) for which regulation in vivo remains largely unknown. The mammalian PI-PLCs are strictly dependent on Ca2<87> and show a clear substrate pre> ference in the order PIP2 > PIP > PI (reviewed by Bruzik & Tsai, 1994).

PI-PLCs with much smaller molecular masses, ranging from 30 kDa to 35 kDa are secreted in large quantities by bacteria such as Bacillus cereus, B. thuringiensis, Staphylococcus aureus, and Listeria monocytogenes (Ikezawa, 1991). They play a role as virulence factors in pathogenic bacteria (Mengaud È et al., 1991; Leimeister-Wachter et al., 1991; Daugherty & Low, 1993), but in most cases their precise physiological function remains elusive. In contrast to their mammalian counterparts, they do not require Ca2<87> for activity. They also do not hydrolyse PIP or PIP2. The cleavage of PI, however, occurs at rates about ten times faster than for mammalian PI-PLCs (Bruzik & Tsai, 1994), about the same as the rate at which eukaryotic PI-PLCs hydrolyse PIP or PIP2 (Ellis & Katan, 1995).

A distinctly different type of eukaryotic phospholipase C is the glycosylphosphatidylinositolspeci®c PLC (GPI-PLC) expressed by the human pathogenic parasite Trypanosoma brucei. Unlike other eukaryotic PI-PLCs, this 39 kDa enzyme behaves like an integral membrane protein (Hereld et al., 1988). In many respects, however, this enzyme is more similar to prokaryotic PI-PLCs than other eukaryotic PI-PLCs. It is metal-independent and hydrolyses the GPI-anchor of variant surface glycoprotein or GPI biosynthetic intermediates


 The comparison of a prokaryotic and eukaryotic PI-PLC structure in combination with multiple sequence alignments suggest a divergent evolution starting from a common ancestor molecule rather than convergent evolution to a stable fold.

An alternative evolutionary scenario can be proposed for bacterial PI-PLCs due to the fact that the natural substrate membrane lipid PI is absent in most bacteria suggesting that most bacteria do not require PI-PLC for their own survival (Camilli et al., 1991). In the case of animal and human pathogens, PI-PLC acts as a virulence factor to facilitate infection of their respective hosts. This is corroborated by the fact that no PI-PLC activity is detected in avirulent Listeria strains (Notermans et al., 1991). It is plausible that bacterial PI-PLCs might be descendants from eukaryotic PI-PLCs that were incorporated by the bacteria during evolution and later changed by purging additional domains and by optimizing the C-terminal half of the barrel.




4
=============================================  REEEEEEEEEEEEEEEADDDDDDD AGAAAINNNNNNNNNNNNN
A large number of extracellular signals stimulate hydrolysis of phosphatidylinositol 4,5-bisphosphate by phosphoinositidespecific phospholipase C (PI-PLC). PI-PLC isozymes have been found in a broad spectrum of organisms and although they have common catalytic properties, their regulation involves different signalling pathways

The hydrolysis of phos- phatidylinositol 4,5-bisphosphate (PtdIns 4,5-P2 ) by a speci¢c phospholipase C (PI-PLC) to two important second messengers, diacylglycerol and inositol 1,4,5-trisphosphate (Ins 1,4,5-P3 ), is one of the earliest key events triggered by a large number of extracellular signalling molecules [1^4].

 Based on the structural similarity and the common catalytic mechanism, it is unlikely that these two groups of PI-PLC represent products of convergent evolution. One possibility could be a divergent evolution from a common ancestor. However, since PI-PLCs are not essential for bacterial survival and are often found in pathogens, they could be descendants from eukaryotic PI-PLCs that were incorporated by the bacteria during evolution.


 Based on comparison of their sequences and structural studies, they can be regarded as a superfamily of related proteins. Within the superfamily of PI-PLCs, bacterial enzymes are clearly a distinct group from the three families of eukaryotic isozymes, PLCL, PLCQ and PLCN.



 5
 =========================

  These studies demonstrate that antibodies raised against bacterial phospholipase C may be 0 1986 Academic Press, Znc. useful in purifying phospholipase C from a human source.



 6
 ==================================
 The physiological function of the PC-PLC is not known, but the enzyme was reported to be part of a phosphate retrieval system (Guddal et al. 1989).

 It is synthesized as a 283-amino acid long pre-proenzyme from which the 24-aa long presequence and the 14-aa long prosequence are cleaved first. The mature enzyme has a length of 245 aa and a calculated molecular weight of 28,520 Da, including the three zinc ions




 7
 =======================

 It has been known for many years that during growth Bacillus cereus secretes three different phospholipase C enzymes, viz., a phosphatidylinositol-hydrolyzing enzyme (PI-PLC), a sphingomyelinase C (SMase C) which also possesses activity hemolytic to mammalian erythrocytes (5), and a phospholipase C which attacks phosphatidylcholine, phosphatidylethanolamine, and phosphatidylserine (PCPLC) (23).

 concluded that B. cereus operates a Pi-repressed phosphorus retrieval-scavenging system.
 
 he marked decreases in APase and SMase activities noted above could be explained by secretion of protease activity from the bacteria. 

  Pseudomonas aeruginosa also secretes a PC-PLC which, together with a heat-stable hemolysin and alkaline phosphatase (APase), seems to be part of a phosphate retrieval system operating under growthlimiting concentrations of Pi
  We studied the production of certain extracellular enzyme activities of B. cereus relative to Pi concentrations in the medium and concluded that B. cereus operates a Pi-repressed phosphorus retrieval-scavenging system.

  Production of PI-PLC exoenzyme activity by B. cereus was not influenced by P1 levels in the medium. However, phosphatidylinositol is a very minor phospholipid in biological membranes and has little potential as a source of phosphorus. Presumably, the physiological function of B. cereus PI-PLC differs from that of the two other phospholipases C.




8
===================================================

These findings suggest that the binding pocket of wild-type PC-PLCBc has evolved to preferentially bind and hydrolyze PC. 
