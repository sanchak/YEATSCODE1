
Our work centres mostly around determining the active residues for the protein {\bf\em uvi31}~\cite{OUR} which shows  {\bf\em $\beta$ lactamase} activity.

$\beta$-Lactamases are the main cause of bacterial resistance to penicillins, 
cephalosporins and related $\beta$-lactam compounds. 
These enzymes inactivate the antibiotics by hydrolysing the amide bond of the $\beta$-lactam ring. 

The active-site serine $\beta$ lactamases are divided into three molecular classes {\bf A}, {\bf C} and {\bf D}. 
A minority of enzymes are metalloproteins whose activity relies on the presence of one or two zinc ion(s) 
in their active site are known as class {\bf B} $\beta$-lactamases~\cite{BLACCLASSIFICATION}.

For example,  several {\em conserved} elements have been identified which appear to be directly or indirectly involved in the substrate recognition and catalytic processes in the neighbourhood of the active-site serine residue. As discussed in detail in~\cite{BLACCLASSIFICATION2} there are the following motifs. 
\begin{itemize}
    \item Active serine and, one helix-turn downstream, a lysine residue (SER-X-X-LYS sequence). 
	\item TYR-X-ASN ($\beta$-lactamases of classes C and D, some PBPs) or SER-X-ASN ($\beta$-lactamases of class A, most PBPs) sequences.
    \item Lys-Thr-Gly sequence, but Lys is replaced by His or Arg in a few proteins~\cite{1G68} and Thr by Ser in several class A $\beta$-lactamases.
	\item The fourth motif present in class A $\beta$-lactamases where the two residues Glu-166 and Asn-170 are critical~\cite{MUTATE1}.
\end{itemize}
	 
Interestingly there are proteins~\cite{1O75} which inspite of exhibiting $\beta$-lactamase activity and possessing the motifs mentioned above, do not lose
their PBP activity when the amino acids within the known motifs are mutated. Thus they seem to define a new class of PBP.
It has also been demonstrated~\cite{HIGHTOLERANCE} proteins are highly tolerant to active site mutations.


To summarize the search for active residues in a protein showing $\beta$-lactamase activity is essentially made non trivial as 
\begin{description}
   \item[Multiple Motifs] There are many motifs with varying degrees of neccessity
   \item[Equivalent Residues] Residues can be replaced with stereochemical \mbox equivalent residues.
   \item[New Classes] Proteins where active residues do not reside in known motifs.
\end{description}


Such variability in $\beta$-lactamases, neccessitate the developments of programs which will be able to query proteins with known enzymatic actions
and unknown active sites using proteins showing similar activities and known active residues. {\em User defined flexibility} is of essence in such 
a system to model the flexibility nature has provided in designing proteins.




