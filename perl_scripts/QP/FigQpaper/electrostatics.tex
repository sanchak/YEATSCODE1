
\clearpage
\pagebreak
\clearpage
\subsubsection{\bf Electrostatics in proteins:}
Debye H$\ddot{u}$ckel formulated a simple model for calculating the chemical potentials due to an ionic species
and other ions. In this model one can consider the solution having charged ions and water molecules, and arbitrarily
choose a reference atom and give it a charge. The remaining ions are considered smeared into a continuous
distribution of charge, leading to a net charge density per unit volume. Electroneutrality requires
that the total charge of the system must be equal and opposite to the charge of the central atom. Water
is considered as a continuous dielectric medium. This reduces the problem of computing the 
distribution of ions inside an electrolytic solution to the simpler problem of 
determining how the charge density varies with distance from the central ion.


The relation between the charge density ($\rho_{r}$) and the electrostatic potential ($\Phi_{r}$) at a point $r$  is given in Equation~\ref{eqn:1} by Poisson's equation ($\epsilon$ is the dielectric constant which is a discontinuous function in this case). The excess charge density can also be written in terms of the number of ionic charges of each species ($n_{i}$) and the charge ($q_{i}$) as in Equation~\ref{eqn:2}.
\begin{align}
\label{eqn:1}      \nabla^{2}\Phi_{r} = - \frac{-4\pi\rho_{r}}{\epsilon}  \\
\label{eqn:2}      \rho_{r} = \sum n_{i}q_{i} 
\end{align}



The Boltzmann distribution of classical statistical physics is shown in Equation~\ref{eqn:3}, where $T$ is the absolute temperature, $k$ is Boltzmann constant and $n_{i}^{o}$ is the bulk concentration of the $i_{th}$ species. 
\begin{align}
\label{eqn:3}  n_{i} = n_{i}^{o}e^{-U/kT} 
\end{align}

If the forces are attractive, then the potential energy $U$ is negative and  $n_{i} > n_{i}^{o}$. This implies that the concentration are more than that of the bulk concentration which is logical since the force is attractive.
For dilute solutions one can make the fundamental assumption that the short range forces are negligible and only coulombic forces are considered, and hence
\begin{align}
\label{eqn:4}  U = q_{r}\Phi_{r}
\end{align}

Using Equation~\ref{eqn:3} and ~\ref{eqn:4} we can write the charge density in Equation~\ref{eqn:2} as
\begin{align}
\label{eqn:5}      \rho_{r} = \sum n_{i}q_{i} = \sum n_{i}^{o}q_{i}e^{-q_{r}\Phi_{r}/kT} 
\end{align}


Another simplifying assumption that the average electrostatic potential is much smaller than the thermal energy $kT$,
allows the Taylor series expansion of the exponential term. Ignoring all terms except the first two, and ensuring 
electrostatic neutrality gives us
\begin{align}
\nonumber      \rho_{r}  & =  \sum n_{i}q_{i} -  \sum \frac{n_{i}^{o}q_{i}^{2}\Phi_{r}}{kT} \\
\label{eqn:6}      & =   -  \sum \frac{n_{i}^{o}q_{i}^{2}\Phi_{r}}{kT}
\end{align}



Thus one has derived two expressions for the charge density in Equations~\ref{eqn:1} and ~\ref{eqn:6}. 
Equating these one obtains the linearized Poisson-Boltzmann equation. In this derivation, it has been assumed that there are no mobile ions.

\begin{align}
\label{eqn:7}      \nabla^{2}\Phi_{r} = \left( \frac{-4\pi}{ \epsilon kT  }  \sum n_{i}^{o}q_{i}^{2} \right)\Phi_{r}  
\end{align}

Assigning charge and radius parameters in a protein~\citep{PDB2PQR} using a force field model, current tools~\citep{DELPHI,APBS} solve for the electrostatic potential using finite difference methods~\citep{FINITEDIFF}. 

