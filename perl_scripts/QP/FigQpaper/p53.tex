1 
=======================================================
The cell death machinery governed by the p53 tumor suppressor in response to DNA damage

The cellular response to genotoxic stress that damages DNA includes cell cycle arrest, activation of DNA repair, and in the event of irreparable damage, induction of apoptosis.

 When cells encounter genotoxic stress, certain sensors for DNA lesions eventually stabilize and activate p53. 


 Therapeutic anti-cancer treatments that use DNA-damaging agents must strike a balance between induction of repair and apoptosis in order to maximize the therapeutic effect. 

The p53 gene is one of the most common sites for genetic alterations in human solid cancers since it is mutated in more than 50\% cancer cases worldwide.

The level of p53 protein is basically undetectable in normal cells but rapidly increases in response to a variety of stress signals. 

Subsequent research on wild-type p53 clearly demonstrates that the transcription factor is a key controller in cell cycle and determines the cell fate in response to oncogenic and to other stresses

The biological role of p53 is to ensure genome integrity of cells.

 The transcriptional activation is determined by the N-terminus of p53,


 The N-terminus is heavily phosphorylated, while the C-terminus contains phosphorylated, acetylated, methylated and sumoylated residues.

Post-translational modifications on the N-terminus are important for stabilizing p53 

C terminal - inhibit the ability for regulation of sequence-specific DNA binding, the oligomerization state, the nuclear import ⁄ export process, and the ubiquitination.(


The mdm2 gene is a transcriptional target of p53, and once synthesized, the MDM2 protein can bind to p53 at the N-terminus leading to its rapid degradation through the ubiquitin-proteasome machinery.
These phopshorylated sites in the N-terminus of p53 are close to the Mdm2-binding region of the protein, thereby they block the interaction with Mdm2, leading to stabilization of p53, that escapes from proteasomal degradation.



p53 activates several important genes that are crucial for the execution of the intrinsic pathway of apoptosis including pro-apoptotic genes such as Bax, Noxa, Puma, and Apaf–1.(1

op53 represses the apoptosis repressor with caspase recruitment domain protein, which counteracts the apoptotic functions of Puma and Bad.(22) p53 can also promote cytochrome c release by inducing the expression of the OKL38 tumor suppressor gene, which localizes to the mitochondria and augments cytochrome c release


regulation of p53 in response to genotoxic stress commonly occurs by hampering ubiquitination and subsequent degradation of the p53 protein. By contrast, suppression of p53 expression by inhibition of PKCd is caused by the inhibition of p53 synthesis, not increased degradation of p53 protein.



 Most important modification for the induction of apoptosis is the phosphorylation at serine 46 
  DYRK2 phosphorylates p53 at Ser46 in cells exposed to genotoxic stress.
  DYRK2 translocates from the cytoplasm into the nucleus in response to DNA damage.


  study demonstrated that the expression of a p53-46F mutant, in which Ser46 is replaced with phenylalanine, induces specific p53-target genes associated with apoptosis, including Noxa, p53AIP1, and p53RFP.


  Lys320 
  i p53 transcriptional activity and growth arrest function is enhanced by acetylation at Lys320 by the acetyltransferase PCAF
  On the other hand, the E3 ligase E4F1 ubiquitinates p53 at Lys320, and specifically increases the activation of cell-cycle-arrest genes, such as p21, Gadd45, and cyclin G1 while the expression of apoptotic targets remains unchanged.
  These results are consistent with a model whereby acetylation at Lys320 and Lys373 exerts as a sensor system that enables p53 to determine between growth arrest and cell death, and to coordinate gene expression patterns appropriately following DNA damage.



. Lys120 lies in the DNA-binding domain of p53, and its acetylation leads to increased recruitment of p53 specifically to pro-apoptotic target genes, such as Puma and Bax, suggesting that this modification alone can influence how p53 responds to the DNA-damaging signals. This modification seems to be required for p53-dependent apoptosis, as mutants that can no longer be modified in this way exhibit impaired apoptotic activity while maintaining the proper regulation of Mdm2 and growth-arrest genes.





2 
=====================================================================================================================

The multiple levels of regulation by p53 ubiquitination


mdm2, a RING oncoprotein, was once thought to be the sole E3 ubiquitin ligase for p53 - 

however recent studies have shown that p53 is stabilized but still degraded in the cells of Mdm2-null mice

. The function of ubiquitinated p53 varies in the nucleus and cytosol underlying the many potential contributions ubiquitinated p53 may have in promoting cell proliferation or death.



Genotoxic stress, among others, induces p53 stabilization and activation setting off processes that result in the expression of various factors contributing to cell cycle arrest and death. On account of the rapid and concise p53 response to stressors, there is a need for tight regulations of p53 expression and activation for cells to respond appropriately to environmental stressors.


oMethylation, in particular, has been associated with both the enhancement and repression of p53 function, dependent on the specific site of methylation

 Polyubiquitination, where four or more ubiquitin monomers are conjugated to a substrate, commonly targets it for proteasomal degradation.

  Mdm2 is a RING finger domain containing protein that exhibits E3 ubiquitin-protein ligase activity and is capable of regulating its own levels through auto-ubiquitination.
   The major lysine residues that are ubiquitinated by Mdm2 have been narrowed down to six lysines in the C-terminal of p53, K370, K372, K373, K381, K382, and K386.
   Conversely, many of the same sites that are ubiquitinated by Mdm2 can also be acetylated by p300/CBP leading to p53 activation
   Furthermore, only the polyubiquitinated form of p53 is associated with p53 destabilization and proteasomal degradation, whereas the mono-ubiquitinated form of p53 is targeted for nuclear export


   he MdmX RING domain does not have E3 ligase activity

   After ubiquitination has served its purpose in targeting p53 for various functions, deubiquitinating enzymes remove ubiquitin from p53. HAUSP (USP7) was identified to deubiquitinate p53 in the nucleus resulting in p53 stability.35 By preventing p53 degradation, there is an increase in cell arrest and apoptosis when HAUSP is overexpressed. HAUSP may also play a role in the cytosol by deubiquitinating monoubiquitinated p53 and affect p53 function.

 As previously discussed, K48-linked ubiquitination targets p53 for degradation

The first report identified an interaction between p53 and BclXL, a Bcl2-family member, at the mitochondria, which promotes cytochrome c release.59 Cytochrome c is released into the mitochondria after mitochondrial membrane integrity is compromised and subsequently activates the caspase cascade leading to apoptosis.
